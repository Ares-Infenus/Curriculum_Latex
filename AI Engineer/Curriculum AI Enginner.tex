\documentclass[11pt,a4paper]{moderncv}

% Estilo del CV (clásico, casual, banking, etc.)
\moderncvstyle{banking}
% Estilo de color (azul, verde, rojo, etc.)
\moderncvcolor{black}

% Configuración de márgenes
\usepackage[scale=0.85]{geometry}

% Configuración de fuente
\usepackage{ebgaramond} % Cambiar la fuente a EB Garamond

% Cargar paquetes para listas personalizadas
\usepackage{enumitem}

% Configuración del entorno itemize
\setlist[itemize]{
  left=2em, % Espaciado inicial
  labelsep=0.5em, % Espaciado entre la viñeta y el texto
  itemsep=0.5em, % Espaciado vertical entre ítems
}
% Información del encabezado
\name{Sebastián David }{Pinzón Zambrano}
\title{AI Engineer.} % Título opcional
\address{Carrera 16 \#20B-53}{Valledupar, Colombia} % Dirección opcional
\phone[mobile]{+57 318 6087347} % Número de teléfono móvil
\email{spinzonzambrano@gmail.com} % Correo electrónico
\social[linkedin]{https://www.linkedin.com/in/sebastiandpinzon/} % LinkedIn opcional
\social[github]{https://github.com/Ares-Infenus} % GitHub opcional

% Inicio del documento
\begin{document}

% Encabezado
\makecvtitle

% Sección: Perfil profesional
\section{Perfil profesional}
Ingeniero de IA con experiencia en análisis de datos, big data y desarrollo de soluciones basadas en aprendizaje automático. Enfocado en la optimización de procesos y la implementación de tecnologías innovadoras para transformar datos en decisiones estratégicas.
% Sección: Habilidades
\section{Habilidades}
\cvitem{Lenguajes de programación}{Python, C++, SQL, R, JavaScript, HTML, CSS, LaTeX.}
\cvitem{Herramientas de análisis}{Excel, Tableau, Scikit-learn, TensorFlow, Pandas, Matplotlib.}
\cvitem{Bases de datos y Big Data}{MongoDB, PostgreSQL, Spark, Hadoop.}
\cvitem{Cloud Computing}{AWS, Google Cloud, Azure.}
\cvitem{Habilidades blandas}{Liderazgo, gestión de proyectos, comunicación efectiva.}

% Sección: Experiencia laboral
\section{Experiencia laboral}
\cventry{2023--2024}{Administrador}{Imperio del plástico}{Valledupar, Colombia}{}%
{
  \begin{itemize} % Viñetas normales
    \item Lideré la implementación de análisis de datos avanzados, utilizando modelos predictivos y algoritmos de optimización para mejorar procesos financieros y operativos, reduciendo costos en un 12\%.
    \item Utilicé análisis cuantitativo y modelado de datos avanzado en Excel para apoyar decisiones estratégicas en contabilidad y finanzas, mejorando la eficiencia en un 63\%.
    \item Introduje tecnologías innovadoras y herramientas de análisis de datos en la gestión financiera y control de inventarios, mejorando la eficiencia operativa en un 20\%.
    \item Cofundé y gestioné una fábrica de parrillas, estableciendo una operación rentable en 3 meses, con un enfoque en la optimización de recursos.
    \item Apliqué un enfoque basado en data-driven decision-making y análisis estadístico que generó un incremento sostenido del 13\% mensual en los ingresos de la empresa.
  \end{itemize}
}

% Sección: Educación
\section{Educación}
\cventry{2019--2023}{Administración de Sistemas Informáticos}{Universidad Nacional de Colombia}{Manizales, Colombia}{}%
{
  Nota: 7 Semestres Completados
}
\cventry{2012--2018}{Bachiller}{Institución Educativa Alfonso López Pumarejo}{Valledupar, Colombia}{}{}

% Sección: Proyectos
\section{Proyectos}
\cvitem{\href{https://github.com/Ares-Infenus/HERMESDB}{HermesDB}}{Este proyecto se enfoca en almacenar y gestionar datos históricos de más de 875 activos financieros, con el objetivo de optimizar la extracción de datos. La infraestructura está basada en Arquitectura Orientada a Servicios (SOA), utilizando Python para el procesamiento de datos y Web Scraping para obtener información financiera adicional. Los datos están almacenados en una base de datos Oracle SQL, fácilmente transferible entre computadoras, y se actualiza automáticamente cada semana.}

% Sección: Certificaciones
\section{Certificaciones}
\cvitem{2024}{\textit{\href{}{Deep Learning Specialization}} --- \textbf{DeepLearning.AI.} }
\cvitem{2025}{\textit{\href{}{Machine Learning Specialization}} --- \textbf{Stanford, DeepLearning.AI.} }
\cvitem{2025}{\textit{\href{}{AWS Cloud Solutions Architect Professional Certificate}} --- \textbf{AWS.} }
\cvitem{2025}{\textit{\href{}{Data Engineering, Big Data, and Machine Learning on GCP Specialization}} --- \textbf{Google} }
\cvitem{2025}{\textit{\href{}{Mathematics for Machine Learning Specialization}} --- \textbf{Imperial College London.} }
\cvitem{2025}{\textit{\href{}{DeepLearning.AI Data Engineering Professional Certificate}} --- \textbf{AWS, DeepLearning.AI} }

% Sección: Reconocimientos y premios
\section{Reconocimientos y premios}
\cvitem{2018}{Mejores pruebas Saber 11.}

% Fin del documento
\end{document}
